\section{Resumen del sistema}

El Sistema de Comisiones es usado para grabar información de las comisiones que realizan los empleados (fechas, tipo de comisión, lugar, comprobantes). Los reportes son generados por el personal del área administrativa y financiera de la empresa, en ellos se incluye información acerca de los empleados de las comisiones, gastos por área, gastos por empleado, etc. \\

Las puntos clave del sistema son los siguientes:

\begin{itemize}
	\item \textit{Solicitud de comisiones} Los empleados pueden solicitar comisiones ya sea de viáticos o transporte, así mismo visualizar el estado de aprobación de estas ya sea si fueron aprobadas o rechazadas.
	
	\item \textit{Aprobación de comisiones} El sistema maneja el flujo de aprobación de comisiones. Para que una comisión sea completamente aprobada esta debe ser aprobada en primera instancia por el jefe de área del departamento al que pertenece cada empleado, posteriormente las comisiones aprobadas por los jefes de área deben ser aprobadas por el área financiera donde esta aprobación se realiza tomando distintos criterios, como el presupuesto de la empresa destinado a cada área o departamento y partida.
	
	\item \textit {Manejo de comprobantes} Los empleados pueden subir comprobantes de gastos de sus comisiones, estos comprobantes, el sistema debe validar que estos archivos son con terminación .xml y así mismo permitir su visualización por parte de los jefes de área y finanzas.
	
	\item \textit {Reportes administrativos y financieros} El sistema generara diversos reportes mostrando el numero de comisiones realizadas por mes, gastos por empleado, área, etc.
\end{itemize}
