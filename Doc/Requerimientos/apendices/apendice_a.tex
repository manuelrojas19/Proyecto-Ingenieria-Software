\chapter{Propiedades no funcionales}
A continuación, se listan las propiedades no funcionales sobre las cuales se trabaja en el presente documento y sus definiciones correspondientes, las cuales fueron tomadas de (Referencia al número de la bibliografía correspondiente al libro de arquitectura de software)

\begin{itemize}
	\item \textbf{Eficiencia} La eficiencia es una calidad que refleja la capacidad de un sistema de software para cumplir con sus requisitos de rendimiento y minimizar el uso de los recursos en su entorno informático. En otras palabras, la eficiencia es una medida de la economía de uso de recursos de un sistema.
	\item \textbf{Complejidad} La complejidad es una propiedad del sistema de software que es proporcional al tamaño del sistema, el número de sus elementos constituyentes, el tamaño y la estructura interna de cada elemento, y el número y la naturaleza de las interdependencias elementales.
	\item \textbf{Escalabilidad} Se divide en:
		\begin{itemize}
			\item \textbf{Escalabilidad} La escalabilidad es la capacidad de un sistema de software para adaptarse a los nuevos requisitos de tamaño y alcance.
			\item \textbf{Heterogeneidad} La heterogeneidad es la calidad de un sistema de software que consta de múltiples componentes dispares o que funciona en múltiples entornos informáticos dispares, además para consistir en múltiples componentes dispares o funcionar en múltiples entornos informáticos dispares.
			\item \textbf{Portabilidad} La portabilidad es la capacidad de un sistema de software para ejecutarse en múltiples plataformas (hardware o software) con modificaciones mínimas y sin degradación significativa en las características funcionales o no funcionales.
		\end{itemize}
	\item \textbf{Adaptabilidad} La adaptabilidad es la capacidad de un sistema de software para satisfacer nuevos requisitos y ajustarse a las nuevas condiciones de funcionamiento durante su vida útil.
	\item \textbf{Confiabilidad} Se divide en:
		\begin{itemize}
			\item \textbf{Fiabilidad} La fiabilidad de un sistema de software es la probabilidad de que el sistema realice su funcionalidad prevista dentro de los límites de diseño especificados, sin fallas, durante un período de tiempo determinado. 
			\item \textbf{Disponibilidad} La disponibilidad de un sistema de software es la probabilidad de que el sistema esté operativo en un momento determinado.
			\item \textbf{Robustez}  Un sistema de software es robusto si puede responder adecuadamente a condiciones de tiempo de ejecución no anticipadas.
			\item \textbf{Tolerancia a fallos} Un sistema de software es tolerante a fallas si puede responder con gracia a las fallas en tiempo de ejecución.
			\item \textbf{Supervivencia} La capacidad de supervivencia es la capacidad de un sistema de software para resistir, reconocer, recuperarse y adaptarse a las amenazas que comprometen la misión.
			\item \textbf{Seguridad en el uso} La seguridad denota la capacidad de un sistema de software para evitar fallas que resultarán en pérdida de vidas, lesiones, daños significativos a la propiedad o destrucción de la propiedad.
		\end{itemize}
\item \textbf{Seguridad} “La protección brindada a un sistema de información automatizado para alcanzar los objetivos aplicables de preservar la integridad, disponibilidad y confidencialidad de los recursos del sistema de información (incluye hardware, software, firmware, información / datos y telecomunicaciones) "(Guttman y Roback 1995).
\end{itemize}

