\section{Requerimientos no funcionales}

	A continuación, se describen los requerimientos no funcionales asociados al proyecto, separados en restricciones de construcción y propiedades no funcionales.


\subsection{Restricciones de construcción}

%Herramientas, diseños, normas, políticas, etc. utilizadas o que se deben respetar
\begin{itemize}
	\item \textbf{RNF-RC1 IDE de desarrollo} El IDE de desarrollo que utilizaremos es eclipse debido a las ventajas que nos ofrece como IDE de desarrollo entre las cuales están: implementación de módulos: lo cual nos permite trabajar de manera fácil con distintas funcionalidades que vayamos requiriendo a la hora del desarrollo
	\item \textbf{RNF-RC2 Lenguaje de programación} El lenguaje de programación que utilizaremos para la construcción del software es Java debido a: al estar desarrollado en java, el sistema fácilmente podrá ejecutarse en múltiples sistemas operativos
	\item \textbf{RNF-                                                                                                                                                                                                                                                                                                                                                                                                                                                                                                RC3 Manejador de Bases de datos} El manejador de bases de datos que utilizaremos es MySQL debido a: es software libre no comercial, lo cual implica una reducción de gastos en el desarrollo del sistema. Disponibilidad en múltiples plataformas y sistemas, lo cual permite que nuestro sistema pueda ejecutarse en múltiples plataformas. Bajo costo en requerimientos para la elaboración de BD, al ser un sistema de gran complejidad este manejador de bases de datos es el indicado dado a que nos garantiza el menor consumo de recursos
	
\end{itemize}

\subsection{Propiedades no funcionales }

%Propiedades no funcionales de software, Eficiencia, Complejidad, etc. y la justificación de si se cumplen o no para el proyecto 
\begin{itemize}
	\item \textbf{RNF-P1 Eficiencia} No se puede cumplir con esta propiedad no funcional debido a que la aplicación no utilizará un nivel de procesamiento y de almacenamiento que se debe cuidar, además de que las funcionalidades están limitadas.
	\item \textbf{RNF-P2 Complejidad} No se puede cumplir con esta propiedad no funcional debido a que no tiene un nivel alto de complejidad ya que es muy simple el proyecto, ya que realiza solo las 4 operaciones (altas, bajas, consultas y actualizaciones). El tamaño del sistema no es muy grande, solo abarca cosas muy generales, como en sus componentes no son numerosos, no utiliza un tamaño de estructura complejo.
	\item \textbf{RNF-P3 Escalabilidad} 
	Se debe cumplir con esta propiedad no funcional debido a que permite ser adaptado para poder agregar nuevos requerimientos:
	\begin{itemize}
		\item Se cumple con el requisito, ya que el programa se divide las funcionalidades en distintos módulos, lo que permite agregar nuevas funcionalidades de manera fácil si es necesario.
	\end{itemize}
	\item \textbf{RNF-P4 Adaptabilidad} Se debe cumplir con esta propiedad no funcional debido a que la aplicación es adaptable, por lo que:
	\begin{itemize}
		\item Las funcionalidades se encuentran en distintos módulos, se tiene una baja interdependencia en los componentes, lo que garantiza la adaptabilidad del sistema.
	\end{itemize}
	
	\item \textbf{RNF-P5 Confiabilidad} Se debe cumplir con esta propiedad no funcional debido a que es confiable la aplicacion, sirve en todas sus funciones dentro de los requerimientos iniciales, no tiene ningún fallo, ya que se le realizo pruebas por cada operación.
\end{itemize}	
