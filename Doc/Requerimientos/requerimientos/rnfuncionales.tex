\section{Requerimientos no funcionales}

A continuación, se describen los requerimientos no funcionales asociados al proyecto, separados en restricciones de construcción y propiedades no funcionales.


\subsection{Restricciones de construcción}

%Herramientas, diseños, normas, políticas, etc. utilizadas o que se deben respetar
\begin{itemize}
	\item \textbf{RNF-RC1 IDE de desarrollo} El IDE de desarrollo que utilizaremos es eclipse debido a las ventajas que nos ofrece como IDE de desarrollo entre las cuales están: implementación de módulos: lo cual nos permite trabajar de manera fácil con distintas funcionalidades que vayamos requiriendo a la hora del desarrollo.
	
	\item \textbf{RNF - RC2 Lenguaje de programación} El lenguaje de programación que utilizaremos para la construcción del software es Java debido a: al estar desarrollado en java, el sistema fácilmente podrá ejecutarse en múltiples sistemas operativos.
	
	\item\textbf{RNF - RC3 Manejador de Bases de datos} El manejador de bases de datos que utilizaremos es MySQL debido a: es software libre no comercial, lo cual implica una reducción de gastos en el desarrollo del sistema. Disponibilidad en múltiples plataformas y sistemas, lo cual permite que nuestro sistema pueda ejecutarse en múltiples plataformas. Bajo costo en requerimientos para la elaboración de BD, al ser un sistema de gran complejidad este manejador de bases de datos es el indicado dado a que nos garantiza el menor consumo de recursos.
	
\end{itemize}

